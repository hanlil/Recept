%%%%%%%%%%%%%%%%%%%%%%%%%%%%%%%%%%%%%%
%------------Julmat------------------%
%%%%%%%%%%%%%%%%%%%%%%%%%%%%%%%%%%%%%%

\section{Julmat}

\clearpage

\subsection{Kryddiga pepparkakor}
\sats{ca 200 kakor}

\begin{table}[H]
	\begin{tabular}{rl}
	% Mängd | Ingrediens (Ev. anmärkning)
	\hline
	&\\
		2$\frac{1}{4}$ dl & \textbf{ljus sirap} \\
		150 g & \textbf{smör} \\
		2$\frac{1}{4}$ dl & \textbf{socker} \\
		$\frac{1}{2}$ & \textbf{ägg} \\
		2 tsk & \textbf{nejlika} (malen) \\
		3 tsk & \textbf{kanel} (malen) \\
		2 tsk & \textbf{ingefära} (malen) \\
		2 tsk & \textbf{kardemumma} (malen) \\
		6$\frac{1}{2}$ - 7$\frac{1}{2}$ dl & \textbf{vetemjöl} \\
	&\\
	\hline
	\end{tabular}
\end{table}

\begin{multicols*}{2}

\noindent \textbf{Tillagning}
\begin{enumerate}
	\itemsep0cm
	\item Koka upp sirapen, låt svalna något
	\item Blanda sirapen med sockret
	\item Tillsätt smöret (rumstempererat)
	\item Tillsätt ägget och kryddorna
	\item Tillsätt mjölet, arbeta degen väl
	\item Låt degen vila svalt till nästa dag
	\item Kavla degen tunt och ta ut kakor
	\item Grädda i 175$\degree$, ca 7 minuter, i mitten av ugnen
\end{enumerate}

\end{multicols*}

\clearpage
%%%%%%%%%%%%%%%%%%%%%%%%%%%%%%%%%%%%%%
%------------Julmat------------------%
%%%%%%%%%%%%%%%%%%%%%%%%%%%%%%%%%%%%%%

\section{Julmat och julbak}

\clearpage

\subsection{Kryddiga pepparkakor}
\sats{ca 200 kakor}

\begin{table}[H]
	\begin{tabular}{rl}
	% Mängd | Ingrediens (Ev. anmärkning)
	\hline
	&\\
		2$\frac{1}{4}$ dl & \textbf{ljus sirap} \\
		150 g & \textbf{smör} \\
		2$\frac{1}{4}$ dl & \textbf{socker} \\
		$\frac{1}{2}$ & \textbf{ägg} \\
		2 tsk & \textbf{nejlika} (malen) \\
		3 tsk & \textbf{kanel} (malen) \\
		2 tsk & \textbf{ingefära} (malen) \\
		2 tsk & \textbf{kardemumma} (malen) \\
		6$\frac{1}{2}$ - 7$\frac{1}{2}$ dl & \textbf{vetemjöl} \\
	&\\
	\hline
	\end{tabular}
\end{table}

\begin{multicols*}{2}

\noindent \textbf{Tillagning}
\begin{enumerate}
	\itemsep0cm
	\item Koka upp sirapen, låt svalna något
	\item Blanda sirapen med sockret
	\item Tillsätt smöret (rumstempererat)
	\item Tillsätt ägget och kryddorna
	\item Tillsätt mjölet, arbeta degen väl
	\item Låt degen vila svalt till nästa dag
	\item Kavla degen tunt och ta ut kakor
	\item Grädda i 175$\degree$, ca 7 minuter, i mitten av ugnen
\end{enumerate}

\end{multicols*}

\clearpage

\subsection{Farfar Bengts saffransbullar}
\sats{36 bullar}

\begin{table}[H]
	\begin{tabular}{rl}
	% Mängd | Ingrediens (Ev. anmärkning)
	\hline
	&\\
		100 g & \textbf{jäst} \\
		200 g & \textbf{smör} \\
		$\frac{1}{2}$ l & \textbf{mjölk} \\
		2 paket (à $\frac{1}{2}$ g) & \textbf{saffran} \\
		1 & \textbf{sockerbit} \\
		1$\frac{1}{2}$ - 2 dl & \textbf{socker} \\
		1 tsk & \textbf{salt} \\
		15 - 17 dl & \textbf{vetemjöl} \\
		1 & \textbf{ägg} \\
		& \textbf{ev. lite mjölk} \\
		& \textbf{russin} \\
	&\\
	\hline
	\end{tabular}
\end{table}

\begin{multicols*}{2}

\noindent \textbf{Tillagning}
\begin{enumerate}
	\itemsep0cm
	\item Smält smöret
	\item Smula gästen i en bunke
	\item Blanda ner mjölken i smöret, värm till ca 37$\degree$
	\item Häll degspadet på jästen, och lös upp jästen
	\item Mortla saffran med sockerbiten, och blanda ner i degspadet
	\item	Blanda i socker och salt
	\item Tillsätt vetemjöl (spara ca 1 dl till utbakningen)
	\item Arbeta degen kraftigt (ca 10 \mbox{minuter}) tills smidig
	\item Låt degen jäsa i 40 minuter \mbox{under} bakduk
	\item Knåda degen, dela i 36 bitar
	\item Rulla degbitarna till finger\-tjocka rullar, snurra till \mbox{lussekatter}
	\item Lägg på plåt och sätt i ev. \mbox{russin}
	\item Låt jäsa i 40 minuter under bakduk
	\item Pensla med ägg och ev. lite mjölk, tryck till russinen
	\item Grädda i 225$\degree$,  6-7 minuter
	\item Låt svalna på galler under bakduk
\end{enumerate}

\end{multicols*}

\clearpage

\subsection{Mormor Monikas mjuka pepparkaka}

\begin{table}[H]
	\begin{tabular}{rl}
	% Mängd | Ingrediens (Ev. anmärkning)
	\hline
	&\\
		3 dl & \textbf{socker}\\
		3$\frac{3}{4}$ dl & \textbf{vetemjöl}\\
		2 msk & \textbf{hasselnötter} (malda)\\
		1 tsk & \textbf{kanel}\\
		1 tsk & \textbf{nejlikor} (malda)\\
		1$\frac{1}{2}$ tsk & \textbf{bikarbonat}\\
		3 dl & \textbf{filmjölk} (3$\%$ fetthalt, rumstempererad)\\
		50 g & \textbf{smör}\\
	&\\
	\hline
	\end{tabular}
\end{table}

\begin{multicols*}{2}

\noindent \textbf{Tillagning}
\begin{enumerate}
	\itemsep0cm
	\item Mal hasselnötterna
	\item Smält smöret
	\item Blanda alla torra ingredienser
	\item Tillsätt rumstempererad filmjölk och avsvalnat smör
	\item Häll smeten i en smord och \mbox{bröad} avlång form
	\item Grädda i 190$\degree$, ca 50 minuter
\end{enumerate}

\end{multicols*}

\clearpage
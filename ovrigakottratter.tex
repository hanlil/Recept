%%%%%%%%%%%%%%%%%%%%%%%%%%%%%%%%%%%%%%
%------------Övriga kötträtter-------%
%%%%%%%%%%%%%%%%%%%%%%%%%%%%%%%%%%%%%%

\section{Övriga kötträtter}

\clearpage

\subsection{Renskav}
\port{5-6}

\begin{table}[H]
	\begin{tabular}{rl}
	% Mängd | Ingrediens (Ev. anmärkning)
	\hline
	&\\
		480 g & \textbf{renskav}\\
		1 & \textbf{gul lök}\\
		1 liten burk & \textbf{champinioner} (skivade)\\
		2$\frac{1}{2}$ dl & \textbf{mellangrädde}\\
		4 msk & \textbf{tomatpuré}\\
		ca $\frac{1}{2}$ tsk & \textbf{salt}\\
		1 krm & \textbf{vitpeppar}\\
	&\\
	\hline
	\end{tabular}
\end{table}

\tillb{ris}

\begin{multicols*}{2}

\noindent \textbf{Tillagning}
\begin{enumerate}
	\itemsep0cm
	\item Tina gärna köttet någon timme före tillagning
	\item Finhacka och fräs löken
	\item Stek köttet
	\item Blanda i svampen
	\item Tillsätt grädde, tomatpuré, salt och peppar
	\item Låt puttra några minuter
\end{enumerate}

\end{multicols*}

\clearpage

\subsection{Viltskav med kantareller}
\port{4-5}

\begin{table}[H]
	\begin{tabular}{rl}
	% Mängd | Ingrediens (Ev. anmärkning)
	\hline
	&\\
		250 g & \textbf{renskav}\\
		1 & \textbf{gul lök}\\
		200 - 300 g & \textbf{kantareller} (frysta)\\
		2-3 dl & \textbf{vispgrädde}\\
		1 tärning & \textbf{köttbuljong}\\
		2 tsk & \textbf{dijonsenap}\\
		1 tsk & \textbf{timjan} (torkad)\\
		3 msk & \textbf{vinbärsgelé}\\
		1 msk & \textbf{persilja} (fryst)\\
		$\frac{1}{2}$ tsk & \textbf{salt}\\
		2 krm & \textbf{svartpeppar}\\
	&\\
	\hline
	\end{tabular}
\end{table}

\tillb{ris}

\begin{multicols*}{2}

\noindent \textbf{Tillagning}
\begin{enumerate}
	\itemsep0cm
	\item Tina gärna köttet och svampen någon timme före tillagning
	\item Finhacka och fräs löken
	\item Fräs svampen
	\item Stek köttet
	\item Blanda lök, kött och svamp
	\item Tillsätt grädde, buljongtäning, senap och timjan
	\item Låt puttra i 5 minuter
	\item Tillsätt gelé och persilja
\end{enumerate}

\end{multicols*}

\clearpage


\subsection{Korvstroganoff}
\port{4}

\begin{table}[H]
	\begin{tabular}{rl}
	% Mängd | Ingrediens (Ev. anmärkning)
	\hline
	&\\
		400 g & \textbf{falukorv}\\
		1 & \textbf{gul lök}\\
		2$\frac{1}{2}$ dl & \textbf{matlagningsgrädde}\\
		$\frac{1}{2}$ dl & \textbf{vatten}\\
		1 msk & \textbf{tomatpuré}\\
		1 tsk & \textbf{soja}\\
		1 msk & \textbf{olja} (till stekning)\\
		lite & \textbf{salt}\\
		lite & \textbf{peppar}\\
		ev. ca $\frac{1}{2}$ msk & \textbf{vetemjöl}\\
	&\\
	\hline
	\end{tabular}
\end{table}

\tillb{ris}

\begin{multicols*}{2}

\noindent \textbf{Tillagning}
\begin{enumerate}
	\itemsep0cm
	\item Strimla falukorven
	\item Finhacka löken
	\item Fräs korven i oljan i några \mbox{minuter}, tillsätt löken och fräs ytterligare någon minut
	\item Tillsätt grädde, tomatpuré och soja
	\item Tillsätt vatten och red av med vetemjöl vid behov
	\item Smaka av med salt och peppar
	\item Låt koka i ca 10 minuter
\end{enumerate}

\end{multicols*}

\clearpage


\subsection{Teriyakikotlett}
\port{4}

\begin{table}[H]
	\begin{tabular}{rl}
	% Mängd | Ingrediens (Ev. anmärkning)
	\hline
	&\\
		600 g & \textbf{kotlett} (i bit)\\
		1 & \textbf{röd lök}\\
		1 & \textbf{röd paprika}\\
		150 ml & \textbf{teriyakisås}\\
		1 - 2 msk & \textbf{gräslök} (fryst)\\
		lite & \textbf{salt}\\
		lite & \textbf{peppar}\\		
	&\\
	\hline
	\end{tabular}
\end{table}

\tillb{äggnudlar}

\begin{multicols*}{2}

\noindent \textbf{Tillagning}
\begin{enumerate}
	\itemsep0cm
	\item Skär kötter i remsor, ca 1 cm breda
	\item Skala, halvera och skiva löken i tunna ringar
	\item Skär paprikan i smala bitar
	\item Stek köttet i ca 3 minuter, salta och peppra
	\item Stek lök och paprika
	\item Blanda kött, lök och paprika
	\item Tillsätt teriyakisåsen och låt det bli varmt
	\item Tillsätt gräslök
\end{enumerate}

\end{multicols*}

\clearpage
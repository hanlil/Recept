%%%%%%%%%%%%%%%%%%%%%%%%%%%%%%%%%%%%%%
%------------Pastasåser--------------%
%%%%%%%%%%%%%%%%%%%%%%%%%%%%%%%%%%%%%%

\section{Pastarätter}

\clearpage

\subsection{Köttfärssås (orginal)}
\port{5 - 6}

\begin{table}[H]
	\begin{tabular}{rl}
	% Mängd | Ingrediens (Ev. anmärkning)
	\hline
	&\\
		1 & \textbf{gul lök} \\
		400 - 500 g & \textbf{köttfärs}\\
		500 g & \textbf{passerade/krossade tomater}\\
		$\frac{1}{2}$ msk & \textbf{vetemjöl}\\
		1 klyfta & \textbf{vitlök} (pressad)\\
		1 rågad tesked & \textbf{köttbuljong}\\
		1 tsk & \textbf{salt}\\
		1 krm & \textbf{vitpeppar}\\
		1 msk & \textbf{tomatpuré}\\
		1 dl & \textbf{vatten}\\
		1 tsk & \textbf{oregano} (torkad) \\
	&\\
	\hline
	\end{tabular}
\end{table}

\tillb{spaghetti}

\begin{multicols*}{2}

\noindent \textbf{Tillagning sås}
\begin{enumerate}
	\itemsep0cm
	\item Finhacka och fräs löken
	\item Bryn köttfärsen
	\item Tillsätt vetemjöl
	\item Tillsätt passerade/krossade tomater
	\item Tillsätt vitlök, buljong och övriga kryddor
	\item Tillsätt vatten vid behov
\end{enumerate}

\end{multicols*}

\clearpage

\subsection{Köttfärssås (Hannas)}
\port{5 - 6}

\begin{table}[H]
	\begin{tabular}{rl}
	% Mängd | Ingrediens (Ev. anmärkning)
	\hline
	&\\
		1 & \textbf{gul lök} \\
		400 - 500 g & \textbf{köttfärs}\\
		500 g & \textbf{passerade tomater}\\
		3 - 4 & \textbf{morötter}\\
		1 klyfta & \textbf{vitlök} (pressad)\\
		1 tärning & \textbf{köttbuljong}\\
		1 tsk & \textbf{salt}\\
		1 krm & \textbf{vitpeppar}\\
		1 msk & \textbf{tomatpuré}\\
		1 tsk & \textbf{oregano} (torkad) \\
	&\\
	\hline
	\end{tabular}
\end{table}

\tillb{pasta}

\begin{multicols*}{2}

\noindent \textbf{Tillagning sås}
\begin{enumerate}
	\itemsep0cm
	\item Finhacka och fräs löken
	\item Grovriv och fräs morötterna
	\item Bryn köttfärsen
	\item Tillsätt passerade tomater
	\item Tillsätt vitlök, buljong och övriga kryddor
	\item Tillsätt vatten vid behov
\end{enumerate}

\end{multicols*}

\clearpage

\subsection{Carbonara}
\port{5}

\begin{table}[H]
	\begin{tabular}{rl}
	% Mängd | Ingrediens (Ev. anmärkning)
	\hline
	&\\
		375 g & \textbf{rimmat (skivat) fläsk}\\
		3 & \textbf{ägg}\\
		$\frac{3}{4}$ dl & \textbf{vispgrädde}\\
		1$\frac{1}{2}$ klyfta & \textbf{vitlök} (finhackad)\\
		$\frac{3}{4}$ tsk & \textbf{salt}\\
		3 krm & \textbf{svartpeppar}\\
		100 g & \textbf{riven ost}\\
	&\\
	\hline
	\end{tabular}
\end{table}

\tillb{spaghetti}

\begin{multicols*}{2}

\noindent \textbf{Tillagning sås}
\begin{enumerate}
	\itemsep0cm
	\item Blanda ägg och grädde
	\item Tillsätt vitlök, salt, peppar
	\item Tillsätt cirka $\frac{3}{4}$ av osten
\end{enumerate}

\noindent \textbf{Tillagning övrigt}
\begin{enumerate}
	\itemsep0cm
	\item Koka pasta
	\item Skär fläsket i småbitar och stek
	\item Häll såsen på pastan och rör om
	\item Tillsätt resten av osten och lite nymalen peppar
	\item Tillsätt fläsket \textit{eller} servera fläsket separat
\end{enumerate}

\end{multicols*}

\clearpage

\subsection{Purjo- och champinionsås}
\port{5}

\begin{table}[H]
	\begin{tabular}{rl}
	% Mängd | Ingrediens (Ev. anmärkning)
	\hline
	&\\
		400 g & \textbf{kyckling} \textit{eller} \textbf{korv}\\
		100 g & \textbf{purjolök}\\
		250 g & \textbf{champinioner} (färska)\\
		1$\frac{1}{2}$ msk & \textbf{vetemjöl}\\
		2$\frac{1}{2}$ dl & \textbf{vispgrädde}\\
		1$\frac{1}{2}$ dl & \textbf{mjölk}\\
		2$\frac{1}{2}$ msk & \textbf{tomatpuré}\\
		$\frac{1}{2}$ tsk & \textbf{salt}\\
		1 krm & \textbf{svartpeppar}\\
		40 g & \textbf{riven ost}\\
	&\\
	\hline
	\end{tabular}
\end{table}

\tillb{pasta}

\begin{multicols*}{2}

\noindent \textbf{Tillagning sås}
\begin{enumerate}
	\itemsep0cm
	\item Skär kycklingen/korven i bitar och tillaga på lämpligt vis
	\item Strimla purjolöken
	\item Dela champinionerna i bitar
	\item Stek champinionerna utan smör
	\item Fräs purjolöken i en gryta eller kastrull
	\item Tillsätt champinionerna och strö på vetemjöl
	\item Tillsätt grädde och mjölk, låt puttra några minuter
	\item Tillsätt tomatpuré, salt, peppar
	\item Tillsätt kycklingen/korven och osten
\end{enumerate}

\end{multicols*}

\clearpage

\subsection{Långkokt älgfärssås}
\port{5}

\begin{table}[H]
	\begin{tabular}{rl}
	% Mängd | Ingrediens (Ev. anmärkning)
	\hline
	&\\
		500 g & \textbf{älgfärs}\\
		1 & \textbf{gul lök}\\
		1 & medelstor \textbf{purjolök}\\
		400 g & \textbf{finkrossade tomater}\\
		1 klyfta & \textbf{vitlök} (hackad)\\
		5 & \textbf{cocktailtomater} (klyftade)\\
		1 msk & \textbf{timjan}\\
		1 msk & \textbf{oregano}\\
		2 msk & \textbf{koncentrerad viltfond}\\
		3 & \textbf{enbär} (krossade)\\
		1 dl & \textbf{vatten}\\
		1 dl & \textbf{vispgrädde}\\
		ca 1 tsk & \textbf{salt}\\
		lite & \textbf{svartpeppar} (nymalen)\\
	&\\
	\hline
	\end{tabular}
\end{table}

\tillb{spaghetti}

\begin{multicols*}{2}

\noindent \textbf{Tillagning sås}
\begin{enumerate}
	\itemsep0cm
	\item Finhacka och fräs löken
	\item Strimla och fräs hälften av purjo\-löken
	\item Bryn älgfärsen
	\item Blanda allt \textit{utom} grädde, salt, peppar och hälften av purjo\-löken i en gryta och låt små\-puttra med lock i några timmar
	\item När en timme återstår tillsätt grädden
	\item Har såsen för mycket vätska, \mbox{koka} ner till önskad konsistens utan lock
	\item Före servering strimla, fräs och tillsätt resten av purjolöken, smaka av med salt och peppar
\end{enumerate}

\end{multicols*}

\clearpage
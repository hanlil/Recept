%%%%%%%%%%%%%%%%%%%%%%%%%%%%%%%%%%%%%%
%------------Efterrätter-------------%
%%%%%%%%%%%%%%%%%%%%%%%%%%%%%%%%%%%%%%

\section{Efterrätter}

\clearpage

\subsection{Professorns chokladdessert}
\port{4}

\begin{table}[H]
	\begin{tabular}{rl}
	% Mängd | Ingrediens (Ev. anmärkning)
	\hline
	&\\
		100 g & \textbf{smör}\\
		100 g & \textbf{mörk choklad} (i bitar)\\
		2 & \textbf{ägg}\\
		1 dl & \textbf{strösocker}\\
		lite & \textbf{florsocker}\\
	&\\
	\hline
	\end{tabular}
\end{table}

\tillb{vispgrädde}

\begin{multicols*}{2}

\noindent \textbf{Tillagning}
\begin{enumerate}
	\itemsep0cm
	\item Smält smöret
	\item Blanda ner smöret i chokladen i smöret och låt den smälta
	\item Vispa ägg och socker pösigt
	\item Tillsätt det avsvalnade chokladsmöret
	\item Fördela smeten i ugnssäkra portions\-formar
	\item Grädda i 250$\degree$, ca 10 minuter, i nedre delen av ugnen
	\item Garnera med florsocker
\end{enumerate}

\end{multicols*}

\clearpage

\subsection{Puffin-muffins \altnamn{Knäckäppelkakor}}
\sats{ca 10 stycken}

\begin{table}[H]
	\begin{tabular}{rl}
	% Mängd | Ingrediens (Ev. anmärkning)
	\hline
	&\\
		1 & \textbf{syrligt äpple}\\
		6 msk & \textbf{smör}\\
		1 & \textbf{ägg}\\
		$\frac{3}{4}$ dl & \textbf{strösocker}\\
		3 msk & \textbf{vispgrädde}\\
		1$\frac{3}{4}$ dl & \textbf{vetemjöl}\\
		$\frac{3}{4}$ tsk & \textbf{bakpulver}\\
		$\frac{3}{4}$ tsk & \textbf{vaniljsocker}\\
		3 msk & \textbf{ljus sirap}\\
		35 g & \textbf{mandelspån}\\
	&\\
	\hline
	\end{tabular}
\end{table}

\tillb{vaniljglass}

\begin{multicols*}{2}

\noindent \textbf{Tillagning}
\begin{enumerate}
	\itemsep0cm
	\item Ställ ut 10 dubbla muffins\-formar på en plåt
	\item Skala och skär äpplet i mycket små tärningar
	\item Smält hälften (3 msk) av \mbox{smöret}
	\item Vispa ägg och socker pösigt
	\item Blanda ner det smälta smöret, grädde, mjöl, bakpulver, vaniljsocker och äppeltärningar
	\item Klicka ut smeten i formarna och grädda i 175$\degree$, ca 15 minuter, i mitten av ugnen
	\item Blanda resten av smöret (3 msk), sirap och mandelspån i en kastrull och koka i ca 3 minuter till simmig konsistens och tills den börjar få lite färg
	\item Föredela mandelknäcken över kakorna och gratinera i 250$\degree$, \mbox{3-5} minuter, tills knäcken lagt sig över kakorna och fått en aning färg
\end{enumerate}

\end{multicols*}

\clearpage

\subsection{Tuijas finska glasstårta}

\begin{table}[H]
	\begin{tabular}{rl}
	% Mängd | Ingrediens (Ev. anmärkning)
	\hline
	&\\
		6 dl & \textbf{vispgrädde}\\
		200 g & \textbf{vit choklad}\\
		2 & \textbf{ägg}\\
		200 g & \textbf{frysta jordgubbar}\\
		1 dl & \textbf{florsocker}\\
		1 msk & \textbf{citronsaft}\\
	&\\
	\hline
	\end{tabular}
\end{table}

\tillb{färska jordgubbar}

\begin{multicols*}{2}

\noindent \textbf{Tillagning}
\begin{enumerate}
	\itemsep0cm
	\item Vispa grädden tills den är fast
	\item Smält chokladen t.ex. över vatten\-bad
	\item Tillsätt äggen ett i taget till chokladen, vispa kraftigt
	\item Tillsätt hälften av grädden
	\item Häll blandningen i form med löstagbar kant
	\item Mixa de frysta jordgubbarna i en matberedare
	\item Tillsätt resten av grädden, florsocker och citronsaft
	\item Bred ut jordgubbsblandningen i formen
	\item Låt stå i frysen minst 4 timmar
	\item Ta fram 30 minuter före \mbox{servering}, garnera med färska jord\-gubbar och/eller hyvlad vit choklad
\end{enumerate}

\end{multicols*}

\clearpage

\subsection{Smulpaj}

\begin{table}[H]
	\begin{tabular}{rl}
	% Mängd | Ingrediens (Ev. anmärkning)
	\hline
	&\\
		3$\frac{1}{2}$ dl & \textbf{vetemjöl}\\
		1$\frac{1}{2}$ dl & \textbf{strösocker}\\
		1$\frac{1}{2}$ dl & \textbf{havregryn}\\
		175-200 g & \textbf{smör/margarin}\\
		& \\
		\hline
		& \\
		1 liter & \textbf{blåbär}\\
		1 dl & \textbf{strösocker}\\
		1 msk & \textbf{potatismjöl}\\
		\textit{eller}& \\
		1 liter & \textbf{färska jordgubbar}\\
		1 dl & \textbf{strösocker}\\
		1 msk & \textbf{potatismjöl}\\
		\textit{eller}& \\
		$\frac{1}{2}$ kg & \textbf{frysta jordgubbar}\\
		1 dl & \textbf{strösocker}\\
		1 msk & \textbf{potatismjöl}\\
		\textit{eller}& \\
		ca 10 & \textbf{äpplen}\\
		2 msk & \textbf{socker}\\
		& \textbf{kanel} \\
	&\\
	\hline
	\end{tabular}
\end{table}

\tillb{vaniljsås}

\begin{multicols*}{2}

\noindent \textbf{Tillagning smuldeg}
\begin{enumerate}
	\itemsep0cm
	\item Blanda mjöl, socker och havregryn
	\item Smula ner smöret/margarinet i torrvarorna	
\end{enumerate}

\noindent \textbf{Tillagning blåbärspaj}
\begin{enumerate}
	\itemsep0cm
	\item Lägg bären i en form
	\item Blanda socker med potatismjöl och strö över bären
	\item Grädda bären i 225$\degree$, ca 10 \mbox{minuter}
	\item Lägg på smuldegen
	\item Grädda frukten i 225$\degree$, 12 \mbox{minuter}
\end{enumerate}
\vfill
\columnbreak

\noindent \textbf{Tillagning jordgubbspaj}
\begin{enumerate}
	\itemsep0cm
	\item Skär de färska eller frysta bären i mindre bitar vid behov
	\item Blanda socker med potatismjöl och strö över bären
	\item Grädda bären i 225$\degree$, ca 10 \mbox{minuter}
	\item Lägg på smuldegen
	\item Grädda frukten i 225$\degree$, ca 12 \mbox{minuter}
\end{enumerate}

\noindent \textbf{Tillagning äppelpaj}
\begin{enumerate}
	\itemsep0cm
	\item Skala och klyfta äpplena
	\item Lägg hälften av frukten i en form
	\item Strö över ca 1 msk socker samt kanel
	\item Lägg på resten av frukten och strö över ytterligare 1 msk socker samt kanel
	\item Grädda frukten i 225$\degree$, ca 10 \mbox{minuter}
	\item Lägg på smuldegen
	\item Grädda frukten i 225$\degree$, ca 12 \mbox{minuter}
\end{enumerate}

\end{multicols*}

\clearpage

\subsection{Knäckäppelpaj}
\sats{4-6 portioner}

\begin{table}[H]
	\begin{tabular}{rl}
	% Mängd | Ingrediens (Ev. anmärkning)
	\hline
	&\\
		2-3 & \textbf{äpplen}\\
		150 g & \textbf{smör}\\
		3 dl & \textbf{havregryn}\\
		1$\frac{1}{2}$ dl & \textbf{socker}\\
		$\frac{1}{2}$ dl & \textbf{ljus sirap}\\
		1$\frac{1}{2}$ dl & \textbf{vetemjöl}\\
		$\frac{1}{2}$ tsk & \textbf{bakpulver}\\
		2 msk & \textbf{mjölk}\\
	&\\
	\hline
	\end{tabular}
\end{table}

\tillb{vaniljsås}

\begin{multicols*}{2}

\noindent \textbf{Tillagning}
\begin{enumerate}
	\itemsep0cm
	\item Smält smöret i en kastrull
	\item Blanda havregryn, socker, vetemjöl och bakpulver
	\item Skala och kärna ur äpplena, skär i tunna skivor
	\item Smörj en glasform och lägg i äpplena
	\item Tillsätt sirap, mjölk och smör till smeten
	\item Fördela smeten över äpplena
	\item Grädda i 175$\degree$, 20-30 \mbox{minuter}
\end{enumerate}

\end{multicols*}

\clearpage
\subsection{Schweizernötkaka}

\begin{table}[H]
	\begin{tabular}{rl}
	% Mängd | Ingrediens (Ev. anmärkning)
	\hline
	&\\
		2 & \textbf{ägg}\\
		3 dl & \textbf{socker}\\
		125 g & \textbf{sötmandel}\\
		1,25 dl & \textbf{vetemjöl}\\
		100 g & \textbf{smält smör}\\
		200 g & \textbf{schweizernötchoklad}\\
		1 dl & \textbf{vispgrädde}\\
		15 g& \textbf{smör}\\
	&\\
	\hline
	\end{tabular}
\end{table}

\tillb{vispgrädde}

\begin{multicols*}{2}

\noindent \textbf{Tillagning}
\begin{enumerate}
	\itemsep0cm
	\item Smält smöret.
	\item Skålla och skala mandeln.
	\item Mal mandeln med mandelkvarn.
	\item Vispa ägg och socker pösigt.
	\item Blanda ner den malda mandeln, mjölet och sist det smälta smöret.
	\item Häll smeten i smord rund kakform med löstagbar kant
	\item Grädda i mitten av ugnen i 175$\degree$, ca 25-30 \mbox{minuter} till lätt gyllenbrun färg.
	\item Låt kakan svalna något.
	\item Smält under tiden chokladen i en kastrull på spisen på låg värme.
	\item Rör ner grädden och sist smöret i chokladen.
	\item Låt chokladsmeten svalan något, men den ska fortfarande vara flytande.
	\item Bred chokladsmeten på kakan och ställ svalt.
\end{enumerate}

\end{multicols*}

\clearpage
%%%%%%%%%%%%%%%%%%%%%%%%%%%%%%%%%%%%%%
%-------Kakor och fikabröd-----------%
%%%%%%%%%%%%%%%%%%%%%%%%%%%%%%%%%%%%%%

\section{Kakor och fikabröd}

\clearpage

\subsection{Mattias kladdkaka}

\begin{table}[H]
	\begin{tabular}{rl}
	% Mängd | Ingrediens (Ev. anmärkning)
	\hline
	&\\
		2 & \textbf{ägg}\\
		3 dl & \textbf{socker}\\
		4 msk & \textbf{kakao}\\
		1$\frac{1}{2}$ dl & \textbf{vetemjöl}\\
		1 krm & \textbf{salt}\\
		150 g & \textbf{smör/margarin} (smält)\\
	&\\
	\hline
	\end{tabular}
\end{table}

\tillb{vispgrädde}

\begin{multicols*}{2}

\noindent \textbf{Tillagning}
\begin{enumerate}
	\itemsep0cm
	\item Smält smör
	\item Rör ihop (vispa ej) ägg och \mbox{socker}
	\item Tillsätt kakao, vetemjöl och salt
	\item Smörj och bröa form med löstagbar kant
	\item Tillsätt resten av smöret till smeten
	\item Grädda i 160$\degree$, 40 - 45 minuter
\end{enumerate}

\end{multicols*}

\clearpage

\subsection{Lingonkaka}

\begin{table}[H]
	\begin{tabular}{rl}
	% Mängd | Ingrediens (Ev. anmärkning)
	\hline
	&\\
		2 & \textbf{ägg}\\
		3 dl & \textbf{socker}\\
		2$\frac{1}{2}$ dl & \textbf{vetemjöl}\\
		$\frac{1}{2}$ tsk & \textbf{bakpulver}\\
		125 g & \textbf{smör/margarin} (smält, avsvalnat)\\
		3-4 dl & \textbf{lingon}\\
	&\\
	\hline
	\end{tabular}
\end{table}

\garn{florsocker}

\begin{multicols*}{2}

\noindent \textbf{Tillagning}
\begin{enumerate}
	\itemsep0cm
	\item Smält smör
	\item Vispa ägg och 2 dl socker pösigt
	\item Tillsätt vetemjöl och bakpulver
	\item Smörj och bröa form med löstagbar kant
	\item Tillsätt resten av smöret till smeten
	\item Häll hälften av smeten i formen
	\item Lägg på lingon och 1 dl socker
	\item Häll på resten av smeten
	\item Grädda i 175$\degree$, 50 minuter
\end{enumerate}

\end{multicols*}

\clearpage

\subsection{Äppelkaka}

\begin{table}[H]
	\begin{tabular}{rl}
	% Mängd | Ingrediens (Ev. anmärkning)
	\hline
	&\\
		2 & \textbf{ägg}\\
		3 dl & \textbf{socker}\\
		3 dl & \textbf{vetemjöl}\\
		100 g & \textbf{smör}\\
		3-4 & \textbf{äpplen}\\
		lite & \textbf{kanel}\\
	&\\
	\hline
	\end{tabular}
\end{table}

\begin{multicols*}{2}

\noindent \textbf{Tillagning}
\begin{enumerate}
	\itemsep0cm
	\item Smörj och bröa form med löstagbar kant
	\item Rör smör och socker pösigt
	\item Blanda i ägg och vetemjöl
	\item Häll smeten i formen
	\item Skala och skär äpplen i tunna klyftor
	\item Vänd klyftorna i socker och \mbox{kanel}, stick ner i smeten
	\item Grädda i 175$\degree$, ca 55 minuter
\end{enumerate}

\end{multicols*}

\clearpage

\subsection{Snittkakor}
\sats{ca 56 kakor}

\begin{table}[H]
	\begin{tabular}{rl}
	% Mängd | Ingrediens (Ev. anmärkning)
	\hline
	&\\
		200 g & \textbf{smör/margarin}\\
		2 dl & \textbf{strösocker}\\
		2 msk & \textbf{vanillinsocker}\\
		2 msk & \textbf{ljus sirap}\\
		2 tsk & \textbf{bakpulver}\\
		ca 5 dl & \textbf{vetemjöl}\\
	&\\
	\hline
	\end{tabular}
\end{table}

\begin{multicols*}{2}

\noindent \textbf{Tillagning}
\begin{enumerate}
	\itemsep0cm
	\item Rör smör och strösocker till en pösig smet
	\item Tillsätt vanillinsocker, sirap, bakpulver och mjöl
	\item Dela degen i fyra bitar och \mbox{platta} ut i längder på pappersklädd plåt
	\item Grädda i 200$\degree$, ca 15 minuter
	\item Skär längderna i smala bitar och låt svalna på plåten
\end{enumerate}

\end{multicols*}

\clearpage

\subsection{Vetebullar}
\sats{ca 40 bullar}

\begin{table}[H]
	\begin{tabular}{rl}
	% Mängd | Ingrediens (Ev. anmärkning)
	\hline
	&\\
		150 g & \textbf{smör/margarin}\\
		5 dl & \textbf{mjölk}\\
		50 g & \textbf{jäst}\\
		1 dl & \textbf{socker}\\
		$\frac{1}{2}$ tsk & \textbf{salt}\\
		2 tsk & \textbf{malen kardemumma}\\
		ca 14 dl & \textbf{vetemjöl}\\
		lite & \textbf{vaniljsocker} eller \textbf{kanel}\\
		1 & \textbf{ägg}\\
		lite & \textbf{pärlsocker}\\
	&\\
	\hline
	\end{tabular}
\end{table}

\begin{multicols*}{2}

\noindent \textbf{Tillagning}
\begin{enumerate}
	\itemsep0cm
	\item Smält smöret/margarinet
	\item Smula gästen i en bunke
	\item Blanda ner mjölken i marga\-rinet, värm till ca 37$\degree$
	\item Häll degspadet på jästen, lös jästen i degspadet
	\item	Blanda i socker, salt och kardemumma
	\item Tillsätt vetemjöl (spara 1 dl till utbakningen)
	\item Arbeta degen kraftigt (ca 10 minuter) tills smidig
	\item Låt degen jäsa i 40 minuter
	\item Knåda degen, dela i fyra delar och kavla ut till avlånga kakor
	\item Bred på margarin, strö på \mbox{socker} och sikta på vaniljsocker eller kanel
	\item Rulla och skär till bullar, sätt på plåt och låt jäsa i 30 \mbox{minuter}
	\item Pensla med ägg och lite mjölk, strö på pärlsocker
	\item Grädda i 250$\degree$, ca 10 minuter
	\item Låt svalna på galler under bakduk
\end{enumerate}

\end{multicols*}

\clearpage

\subsection{Mormor Monikas mjuka pepparkaka}

\begin{table}[H]
	\begin{tabular}{rl}
	% Mängd | Ingrediens (Ev. anmärkning)
	\hline
	&\\
		3 dl & \textbf{socker}\\
		3$\frac{3}{4}$ dl & \textbf{vetemjöl}\\
		2 msk & \textbf{hasselnötter} (malda)\\
		1 tsk & \textbf{kanel}\\
		1 tsk & \textbf{nejlikor} (malda)\\
		1$\frac{1}{2}$ tsk & \textbf{bikarbonat}\\
		3 dl & \textbf{filmjölk} (3$\%$ fetthalt, rumstempererad)\\
		50 g & \textbf{smör}\\
	&\\
	\hline
	\end{tabular}
\end{table}

\begin{multicols*}{2}

\noindent \textbf{Tillagning}
\begin{enumerate}
	\itemsep0cm
	\item Mal hasselnötterna
	\item Smält smöret
	\item Blanda alla torra ingredienser
	\item Tillsätt rumstempererad filmjölk och avsvalnat smör
	\item Häll smeten i en smord och \mbox{bröad} avlång form
	\item Grädda i 190$\degree$, ca 50 minuter
\end{enumerate}

\end{multicols*}

\clearpage
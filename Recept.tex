\documentclass[a4paper,12pt]{article}

%%%%%%%%%%%%%%%%%%%%%%%%%%%%%%%%%%%%%%
%------------Language----------------%
%%%%%%%%%%%%%%%%%%%%%%%%%%%%%%%%%%%%%%
\usepackage[swedish]{babel}
\usepackage[utf8]{inputenc}
\usepackage[T1]{fontenc}

%%%%%%%%%%%%%%%%%%%%%%%%%%%%%%%%%%%%%%
%------------Graphics----------------%
%%%%%%%%%%%%%%%%%%%%%%%%%%%%%%%%%%%%%%
\usepackage{float}
\usepackage{gensymb}
\usepackage{epsfig}
\usepackage{graphics}
\usepackage{fancyhdr}
\graphicspath{{pictures/}}

%%%%%%%%%%%%%%%%%%%%%%%%%%%%%%%%%%%%%%
%------------Structure---------------%
%%%%%%%%%%%%%%%%%%%%%%%%%%%%%%%%%%%%%%
\usepackage{hyperref} %För länkar i PDFen
\usepackage{multicol}
\usepackage{titlesec}
\titleformat{\section}[block]{\Large\bfseries\filcenter}{}{1em}{}
%\titleformat{\subsection}[hang]{\bfseries}{}{1em}{}
\setcounter{secnumdepth}{0}
\setlength{\tabcolsep}{2pt}

%%%%%%%%%%%%%%%%%%%%%%%%%%%%%%%%%%%%%%
%------------Commands----------------%
%%%%%%%%%%%%%%%%%%%%%%%%%%%%%%%%%%%%%%
\newcommand{\altnamn}[1]{\noindent \small{(\textit{eller} #1)}}
\newcommand{\port}[1]{\noindent \textit{#1 portioner}}
\newcommand{\tillb}[1]{\noindent \textit{Serveras förslagsvis med \textbf{#1}} \vspace{1cm}}
\newcommand{\garn}[1]{\noindent \textit{Garnera gärna med \textbf{#1}}}
\newcommand{\sats}[1]{\noindent \textit{En sats motsvarar #1}}


\title{\Huge{Recept}}
\author{\begin{tabular}{c}
Sammanställning av\\
Hanna\\
\\
Senast uppdaterad\\
\today\\
\end{tabular}}
\date{} %Låt stå, förhindrar dubbla datum

\begin{document}

\maketitle

\clearpage

\tableofcontents

\clearpage

%%%%%%%%%%%%%%%%%%%%%%%%%%%%%%%%%%%%%%%%%%%%%%%%%%%%%%%%%%%%%
%%%%%%%%%%%%%%%%%%%%%%%%%%%%%%%%%%%%%%%%%%%%%%%%%%%%%%%%%%%%%

%%%%%%%%%%%%%%%%%%%%%%%%%%%%%%%%%%%%%%
%------------Pastasåser--------------%
%%%%%%%%%%%%%%%%%%%%%%%%%%%%%%%%%%%%%%

\section{Pastarätter}

\clearpage

\subsection{Köttfärssås}
\port{5 - 6}

\begin{table}[H]
	\begin{tabular}{rl}
	% Mängd | Ingrediens (Ev. anmärkning)
	\hline
	&\\
		1 & \textbf{gul lök} \\
		400 - 500 g & \textbf{köttfärs}\\
		500 g & \textbf{passerade tomater}\\
		$\frac{1}{2}$ msk & \textbf{vetemjöl}\\
		1 klyfta & \textbf{vitlök} (pressad)\\
		1 rågad tesked & \textbf{köttbuljong}\\
		1 tsk & \textbf{salt}\\
		1 krm & \textbf{vitpeppar}\\
		1 msk & \textbf{tomatpuré}\\
		1 dl & \textbf{vatten}\\
		1 tsk & \textbf{oregano} (torkad) \\
	&\\
	\hline
	\end{tabular}
\end{table}

\tillb{spaghetti}

\begin{multicols*}{2}

\noindent \textbf{Tillagning sås}
\begin{enumerate}
	\itemsep0cm
	\item Fräs löken
	\item Bryn köttfärsen
	\item Tillsätt vetemjöl
	\item Tillsätt passerade tomater
	\item Tillsätt vitlök, buljong och övriga kryddor
	\item Tillsätt vatten vid behov
\end{enumerate}

\end{multicols*}

\clearpage

\subsection{Carbonara}
\port{5}

\begin{table}[H]
	\begin{tabular}{rl}
	% Mängd | Ingrediens (Ev. anmärkning)
	\hline
	&\\
		375 g & \textbf{rimmat (skivat) fläsk}\\
		3 & \textbf{ägg}\\
		$\frac{3}{4}$ dl & \textbf{vispgrädde}\\
		1$\frac{1}{2}$ klyfta & \textbf{vitlök} (finhackad)\\
		$\frac{3}{4}$ tsk & \textbf{salt}\\
		3 krm & \textbf{svartpeppar}\\
		100 g & \textbf{riven ost}\\
	&\\
	\hline
	\end{tabular}
\end{table}

\tillb{spaghetti}

\begin{multicols*}{2}

\noindent \textbf{Tillagning sås}
\begin{enumerate}
	\itemsep0cm
	\item Blanda ägg och grädde
	\item Tillsätt vitlök, salt, peppar
	\item Tillsätt cirka $\frac{3}{4}$ av osten
\end{enumerate}

\noindent \textbf{Tillagning övrigt}
\begin{enumerate}
	\itemsep0cm
	\item Koka pasta
	\item Skär fläsket i småbitar och stek
	\item Häll såsen på pastan och rör om
	\item Tillsätt resten av osten och lite nymalen peppar
	\item Tillsätt fläsket \textit{eller} servera fläsket separat
\end{enumerate}

\end{multicols*}

\clearpage

%%%%%%%%%%%%%%%%%%%%%%%%%%%%%%%%%%%%%%
%------------Kycklingrätter----------%
%%%%%%%%%%%%%%%%%%%%%%%%%%%%%%%%%%%%%%

\section{Kycklingrätter}

\clearpage

\subsection{Oreganokyckling}
\port{4}

\begin{table}[H]
	\begin{tabular}{rl}
	% Mängd | Ingrediens (Ev. anmärkning)
	\hline
	&\\
		4 & \textbf{kycklingfiléer}\\
		2 klyftor & \textbf{vitlök} (hackad)\\
		1 msk & \textbf{smör}\\
		$\frac{1}{2}$ tsk & \textbf{salt}\\
		1 krm & \textbf{svartpeppar}\\
		2$\frac{1}{2}$ dl & \textbf{matlagningsgrädde}\\
		1 tärning & \textbf{hönsbuljong}\\
		2-3 tsk & \textbf{oregano} (färsk eller torkad)\\
		1$\frac{1}{2}$ msk & \textbf{äppelcider-/balsamvinäger}\\
	&\\
	\hline
	\end{tabular}
\end{table}

\tillb{klyftpotatis}

\begin{multicols*}{2}

\noindent \textbf{Tillagning}
\begin{enumerate}
	\itemsep0cm
	\item Skär kycklingfiléerna på \mbox{längden} i lagom stora bitar
	\item Bryn kycklingen i stekpanna, salta och peppra
	\item Blanda grädde, buljong, äppelcider-/balsamvinäger och oregano i en gryta
	\item Tillsätt kycklingen och vitlöken
	\item Låt koka under lock ca 5 \mbox{minuter}
\end{enumerate}

\end{multicols*}

\clearpage

\subsection{Viltkyckling}
\port{4}

\begin{table}[H]
	\begin{tabular}{rl}
	% Mängd | Ingrediens (Ev. anmärkning)
	\hline
	&\\
		4 & \textbf{kycklingfiléer}\\
		3 msk & \textbf{margarin}\\
		1 krm & \textbf{svartpeppar}\\
		100 g & \textbf{champinjoner}\\
		2$\frac{1}{2}$ dl & \textbf{matlagningsgrädde}\\
		$\frac{1}{2}$ dl & \textbf{svartvinbärssaft}\\
		1 dl & \textbf{vatten}\\
		1$\frac{1}{2}$ msk & \textbf{viltfond}\\
		5 & \textbf{enbär}\\
		1 tsk & \textbf{timjan}\\
		1 msk & \textbf{soja}\\
		1$\frac{1}{2}$ msk & \textbf{vetemjöl}\\
	&\\
	\hline
	\end{tabular}
\end{table}

\tillb{klyftpotatis}

\begin{multicols*}{2}

\noindent \textbf{Tillagning}
\begin{enumerate}
	\itemsep0cm
	\item Skär kycklingfiléerna på \mbox{längden} i lagom stora bitar
	\item Bryn kycklingen i stekpanna, salta och peppra
	\item Skiva champinjonerna och stek i margarin
	\item Rör ner vetemjöl
	\item Blanda i grädde, saft, vatten, fond, enbär, timjan och soja
	\item Tillsätt kycklingen
	\item Låt koka under lock ca 10 \mbox{minuter}
	\item Smaka av med salt och peppar
\end{enumerate}

\end{multicols*}

\clearpage

\subsection{Kycklingwok}
\port{ca 8}

\begin{table}[H]
	\begin{tabular}{rl}
	% Mängd | Ingrediens (Ev. anmärkning)
	\hline
	&\\
		1 kg & \textbf{kycklingkött}\\
		2$\frac{2}{3}$ msk & \textbf{soja}\\
		4 tsk & \textbf{ingefära} (riven)\\
		4 klyftor & \textbf{vitlök} (pressad)\\
		2 & \textbf{paprikor}\\
		2 - 3 & \textbf{palsternackor}\\
		2 - 3 & \textbf{morötter}\\
		150 g (ca 30 cm) & \textbf{purjolök}\\
		2$\frac{1}{2}$ msk & \textbf{sesamfrö}\\
		2 msk & \textbf{sweet chilisås}\\
		2 krm & \textbf{salt}\\
	&\\
	\hline
	\end{tabular}
\end{table}

\tillb{ris}

\begin{multicols*}{2}

\noindent \textbf{Tillagning}
\begin{enumerate}
	\itemsep0cm
	\item Skär kycklingköttet i strimlor eller små bitar
	\item Blanda soja, ingefära och vitlök och häll blandingen över köttet
	\item Skala morötter och palster\-nackor, skär i tunna strimlor
	\item Rensa paprikan och skär i \mbox{tunna} bitar
	\item Skär purjolöken i tunna skivor
	\item Rosta sesamfröna i stekpanna på hög värme (utan matfett) tills de fått lite färg, rör hela \mbox{tiden}
	\item Woka köttet tills det är genomstekt och flytta det sedan till stekgryta
	\item Woka palsternackan, \mbox{morötterna}, purjolöken samt paprikan och blanda med kötttet
	\item Tillsätt salt, sweet chilisås och de rostade sesamfröna
\end{enumerate}

\end{multicols*}

\clearpage

\subsection{Flygande Jacob}
\port{6}

\begin{table}[H]
	\begin{tabular}{rl}
	% Mängd | Ingrediens (Ev. anmärkning)
	\hline
	&\\
		800-900 g & \textbf{kycklingkött}\\
		140 g (1 paket) & \textbf{bacon}\\
		1-2 & \textbf{bananer}\\
		1 dl & \textbf{jordnötter}\\
		2 dl & \textbf{vispgrädde}\\
		3 msk & \textbf{chilisås}\\
	&\\
	\hline
	\end{tabular}
\end{table}

\tillb{ris}

\begin{multicols*}{2}

\noindent \textbf{Tillagning}
\begin{enumerate}
	\itemsep0cm
	\item Stek kycklingen i ugn, 225$\degree$, ca 25 minuter
	\item Skär kycklingen i mindre bitar
	\item Klipp bacon i mindre bitar och stek i stekpanna
	\item Lägg kyckling och bacon i en ugnsfrom, strö över \mbox{jordnötterna}
	\item Skiva bananerna, lägg i formen eller servera färsk som tillbehör
	\item Vispa grädden relativt hårt
	\item Blanda ner chilisåsen i grädden
	\item Bre ut gräddblandningen i \mbox{formen}
	\item Gratinera i ugnen 225$\degree$, ca 10 minuter
\end{enumerate}

\end{multicols*}

\clearpage

%%%%%%%%%%%%%%%%%%%%%%%%%%%%%%%%%%%%%%
%------------Fläskfilérätter---------%
%%%%%%%%%%%%%%%%%%%%%%%%%%%%%%%%%%%%%%

\section{Fläskfilérätter}

\clearpage

\subsection{Fläskfilé med mandeltäcke}
\port{4}

\begin{table}[H]
	\begin{tabular}{rl}
	% Mängd | Ingrediens (Ev. anmärkning)
	\hline
	&\\
		600 g & \textbf{fläskfilé}\\
		50 g & \textbf{smör}\\
		1 klyfta & \textbf{vitlök} (pressad)\\
		1 tsk & \textbf{senap}\\
		1 msk & \textbf{persilja} (finhackad)\\
		1 msk & \textbf{ströbröd}\\
		1 dl & \textbf{flagad mandel}\\
	&\\
	\hline
	\end{tabular}
\end{table}

\tillb{potatis}

\begin{multicols*}{2}

\noindent \textbf{Tillagning}
\begin{enumerate}
	\itemsep0cm
	\item Putsa filén och skär i ca 3 cm tjocka skivor
	\item Platta skivorna något, bryn runt om i stekpanna, salta och peppra
	\item Lägg köttet i en ugnsform
	\item Smält smöret
	\item Blanda ner vitlök, senap, \mbox{persilja}, ströbröd och mandel
	\item Fördela blandningen över \mbox{köttet}
	\item Gratinera köttet i 225$\degree$, ca 10 minuter
\end{enumerate}

\end{multicols*}

\clearpage

\subsection{Fläskfilé med kantarellsås}
\port{4}

\begin{table}[H]
	\begin{tabular}{rl}
	% Mängd | Ingrediens (Ev. anmärkning)
	\hline
	&\\
		600 g & \textbf{fläskfilé}\\
		2 burkar (à 190 g) & \textbf{kantareller}\\
		3 msk & \textbf{smör}\\
		1 & \textbf{gul lök} (finhackad)\\
		$\frac{1}{2}$ tsk & \textbf{salt}\\
		1 krm & \textbf{peppar} (nymald)\\
		1 krm & \textbf{timjan}\\
		2-3 dl & \textbf{vispgrädde}\\
		1 tsk & \textbf{soja}\\
	&\\
	\hline
	\end{tabular}
\end{table}

\tillb{bandspaghetti}

\begin{multicols*}{2}

\noindent \textbf{Tillagning kött}
\begin{enumerate}
	\itemsep0cm
	\item Putsa filén och skär i ca 3 cm tjocka skivor
	\item Platta skivorna något, bryn runt om i stekpanna, salta och peppra
\end{enumerate}

\noindent \textbf{Tillagning sås}
\begin{enumerate}
	\itemsep0cm
	\item Låt svampen rinna av och fräs i hälften av smöret tills all vätska kokat in
	\item Tillsätt resten av smöret, lök, salt, peppar och timjan
	\item Låt allt fräsa ytterligare någon minut
	\item Tillsätt grädde och soja
	\item Låt koka några minuter
\end{enumerate}

\end{multicols*}

\clearpage

%%%%%%%%%%%%%%%%%%%%%%%%%%%%%%%%%%%%%%
%------------Övriga kötträtter-------%
%%%%%%%%%%%%%%%%%%%%%%%%%%%%%%%%%%%%%%

\section{Övriga kötträtter}

\clearpage

\subsection{Renskav}
\port{5-6}

\begin{table}[H]
	\begin{tabular}{rl}
	% Mängd | Ingrediens (Ev. anmärkning)
	\hline
	&\\
		480 g & \textbf{renskav}\\
		1 & \textbf{gul lök}\\
		1 liten burk & \textbf{champinioner} (skivade)\\
		2$\frac{1}{2}$ dl & \textbf{mellangrädde}\\
		4 msk & \textbf{tomatpuré}\\
		ca $\frac{1}{2}$ tsk & \textbf{salt}\\
		1 krm & \textbf{vitpeppar}\\
	&\\
	\hline
	\end{tabular}
\end{table}

\tillb{ris}

\begin{multicols*}{2}

\noindent \textbf{Tillagning}
\begin{enumerate}
	\itemsep0cm
	\item Tina gärna köttet någon timme före tillagning
	\item Finhacka och fräs löken
	\item Stek köttet
	\item Blanda i svampen
	\item Tillsätt grädde, tomatpuré, salt och peppar
	\item Låt puttra några minuter
\end{enumerate}

\end{multicols*}

\clearpage

\subsection{Korvstroganoff}
\port{4}

\begin{table}[H]
	\begin{tabular}{rl}
	% Mängd | Ingrediens (Ev. anmärkning)
	\hline
	&\\
		400 g & \textbf{falukorv}\\
		1 & \textbf{gul lök}\\
		2$\frac{1}{2}$ dl & \textbf{matlagningsgrädde}\\
		$\frac{1}{2}$ dl & \textbf{vatten}\\
		1 msk & \textbf{tomatpuré}\\
		1 tsk & \textbf{soja}\\
		1 msk & \textbf{olja} (till stekning)\\
		lite & \textbf{salt}\\
		lite & \textbf{peppar}\\
		ev. ca $\frac{1}{2}$ msk & \textbf{vetemjöl}\\
	&\\
	\hline
	\end{tabular}
\end{table}

\tillb{ris}

\begin{multicols*}{2}

\noindent \textbf{Tillagning}
\begin{enumerate}
	\itemsep0cm
	\item Strimla falukorven
	\item Finhacka löken
	\item Fräs korven i oljan i några \mbox{minuter}, tillsätt löken och fräs ytterligare någon minut
	\item Tillsätt grädde, tomatpuré och soja
	\item Tillsätt vatten och red av med vetemjöl vid behov
	\item Smaka av med salt och peppar
	\item Låt koka i ca 10 minuter
\end{enumerate}

\end{multicols*}

\clearpage

%%%%%%%%%%%%%%%%%%%%%%%%%%%%%%%%%%%%%%
%------------Julmat------------------%
%%%%%%%%%%%%%%%%%%%%%%%%%%%%%%%%%%%%%%

\section{Julmat}

\clearpage

\subsection{Kryddiga pepparkakor}
\sats{ca 200 kakor}

\begin{table}[H]
	\begin{tabular}{rl}
	% Mängd | Ingrediens (Ev. anmärkning)
	\hline
	&\\
		2$\frac{1}{4}$ dl & \textbf{ljus sirap} \\
		150 g & \textbf{smör} \\
		2$\frac{1}{4}$ dl & \textbf{socker} \\
		$\frac{1}{2}$ & \textbf{ägg} \\
		2 tsk & \textbf{nejlika} (malen) \\
		3 tsk & \textbf{kanel} (malen) \\
		2 tsk & \textbf{ingefära} (malen) \\
		2 tsk & \textbf{kardemumma} (malen) \\
		6$\frac{1}{2}$ - 7$\frac{1}{2}$ dl & \textbf{vetemjöl} \\
	&\\
	\hline
	\end{tabular}
\end{table}

\begin{multicols*}{2}

\noindent \textbf{Tillagning}
\begin{enumerate}
	\itemsep0cm
	\item Koka upp sirapen, låt svalna något
	\item Blanda sirapen med sockret
	\item Tillsätt smöret (rumstempererat)
	\item Tillsätt ägget och kryddorna
	\item Tillsätt mjölet, arbeta degen väl
	\item Låt degen vila svalt till nästa dag
	\item Kavla degen tunt och ta ut kakor
	\item Grädda i 175$\degree$, ca 7 minuter, i mitten av ugnen
\end{enumerate}

\end{multicols*}

\clearpage

\subsection{Sill}
\clearpage

%%%%%%%%%%%%%%%%%%%%%%%%%%%%%%%%%%%%%%
%------------Efterrätter-------------%
%%%%%%%%%%%%%%%%%%%%%%%%%%%%%%%%%%%%%%

\section{Efterrätter}

\clearpage

\subsection{Professorns chokladdessert}
\port{4}

\begin{table}[H]
	\begin{tabular}{rl}
	% Mängd | Ingrediens (Ev. anmärkning)
	\hline
	&\\
		100 g & \textbf{smör}\\
		100 g & \textbf{mörk choklad} (i bitar)\\
		2 & \textbf{ägg}\\
		1 dl & \textbf{strösocker}\\
		lite & \textbf{florsocker}\\
	&\\
	\hline
	\end{tabular}
\end{table}

\tillb{vispgrädde}

\begin{multicols*}{2}

\noindent \textbf{Tillagning}
\begin{enumerate}
	\itemsep0cm
	\item Smält smöret
	\item Blanda ner smöret i chokladen i smöret och låt den smälta
	\item Vispa ägg och socker pösigt
	\item Tillsätt det avsvalnade chokladsmöret
	\item Fördela smeten i ugnssäkra portions\-formar
	\item Grädda i 250$\degree$, ca 10 minuter, i nedre delen av ugnen
	\item Garnera med florsocker
\end{enumerate}

\end{multicols*}

\clearpage

\subsection{Puffin-muffins \altnamn{Knäckäppelkakor}}
\sats{ca 10 stycken}

\begin{table}[H]
	\begin{tabular}{rl}
	% Mängd | Ingrediens (Ev. anmärkning)
	\hline
	&\\
		1 & \textbf{syrligt äpple}\\
		6 msk & \textbf{smör}\\
		1 & \textbf{ägg}\\
		$\frac{3}{4}$ dl & \textbf{strösocker}\\
		3 msk & \textbf{vispgrädde}\\
		1$\frac{3}{4}$ dl & \textbf{vetemjöl}\\
		$\frac{3}{4}$ tsk & \textbf{bakpulver}\\
		$\frac{3}{4}$ tsk & \textbf{vaniljsocker}\\
		3 msk & \textbf{ljus sirap}\\
		35 g & \textbf{mandelspån}\\
	&\\
	\hline
	\end{tabular}
\end{table}

\tillb{vaniljglass}

\begin{multicols*}{2}

\noindent \textbf{Tillagning}
\begin{enumerate}
	\itemsep0cm
	\item Ställ ut 10 dubbla muffins\-formar på en plåt
	\item Skala och skär äpplet i mycket små tärningar
	\item Smält hälften (3 msk) av \mbox{smöret}
	\item Vispa ägg och socker pösigt
	\item Blanda ner det smälta smöret, grädde, mjöl, bakpulver, vaniljsocker och äppeltärningar
	\item Klicka ut smeten i formarna och grädda i 175$\degree$, ca 15 minuter, i mitten av ugnen
	\item Blanda resten av smöret (3 msk), sirap och mandelspån i en kastrull och koka i ca 3 minuter till simmig konsistens och tills den börjar få lite färg
	\item Föredela mandelknäcken över kakorna och gratinera i 250$\degree$, \mbox{3-5} minuter, tills knäcken lagt sig över kakorna och fått en aning färg
\end{enumerate}

\end{multicols*}

\clearpage

\subsection{Tuijas finska glasstårta}

\begin{table}[H]
	\begin{tabular}{rl}
	% Mängd | Ingrediens (Ev. anmärkning)
	\hline
	&\\
		6 dl & \textbf{vispgrädde}\\
		200 g & \textbf{vit choklad}\\
		2 & \textbf{ägg}\\
		200 g & \textbf{frysta jordgubbar}\\
		1 dl & \textbf{florsocker}\\
		1 msk & \textbf{citronsaft}\\
	&\\
	\hline
	\end{tabular}
\end{table}

\tillb{färska jordgubbar}

\begin{multicols*}{2}

\noindent \textbf{Tillagning}
\begin{enumerate}
	\itemsep0cm
	\item Vispa grädden tills den är fast
	\item Smält chokladen t.ex. över vatten\-bad
	\item Tillsätt äggen ett i taget till chokladen, vispa kraftigt
	\item Tillsätt hälften av grädden
	\item Häll blandningen i form med löstagbar kant
	\item Mixa de frysta jordgubbarna i en matberedare
	\item Tillsätt resten av grädden, florsocker och citronsaft
	\item Bred ut jordgubbsblandningen i formen
	\item Låt stå i frysen minst 4 timmar
	\item Ta fram 30 minuter före \mbox{servering}, garnera med färska jord\-gubbar och/eller hyvlad vit choklad
\end{enumerate}

\end{multicols*}

\clearpage

\subsection{Smulpaj}

\begin{table}[H]
	\begin{tabular}{rl}
	% Mängd | Ingrediens (Ev. anmärkning)
	\hline
	&\\
		3$\frac{1}{2}$ dl & \textbf{vetemjöl}\\
		1$\frac{1}{2}$ dl & \textbf{strösocker}\\
		1$\frac{1}{2}$ dl & \textbf{havregryn}\\
		175-200 g & \textbf{smör/margarin}\\
		& \\
		\hline
		& \\
		1 liter & \textbf{blåbär}\\
		1 dl & \textbf{strösocker}\\
		1 msk & \textbf{potatismjöl}\\
		\textit{eller}& \\
		1 liter & \textbf{färska jordgubbar}\\
		1 dl & \textbf{strösocker}\\
		1 msk & \textbf{potatismjöl}\\
		\textit{eller}& \\
		$\frac{1}{2}$ kg & \textbf{frysta jordgubbar}\\
		1 dl & \textbf{strösocker}\\
		1 msk & \textbf{potatismjöl}\\
		\textit{eller}& \\
		ca 10 & \textbf{äpplen}\\
		2 msk & \textbf{socker}\\
		& \textbf{kanel} \\
	&\\
	\hline
	\end{tabular}
\end{table}

\tillb{vaniljsås}

\begin{multicols*}{2}

\noindent \textbf{Tillagning smuldeg}
\begin{enumerate}
	\itemsep0cm
	\item Blanda mjöl, socker och havregryn
	\item Smula ner smöret/margarinet i torrvarorna	
\end{enumerate}

\noindent \textbf{Tillagning blåbärspaj}
\begin{enumerate}
	\itemsep0cm
	\item Lägg bären i en form
	\item Blanda socker med potatismjöl och strö över bären
	\item Grädda bären i 225$\degree$, ca 10 \mbox{minuter}
	\item Lägg på smuldegen
	\item Grädda frukten i 225$\degree$, 12 \mbox{minuter}
\end{enumerate}
\vfill
\columnbreak

\noindent \textbf{Tillagning jordgubbspaj}
\begin{enumerate}
	\itemsep0cm
	\item Skär de färska eller frysta bären i mindre bitar vid behov
	\item Blanda socker med potatismjöl och strö över bären
	\item Grädda bären i 225$\degree$, ca 10 \mbox{minuter}
	\item Lägg på smuldegen
	\item Grädda frukten i 225$\degree$, ca 12 \mbox{minuter}
\end{enumerate}

\noindent \textbf{Tillagning äppelpaj}
\begin{enumerate}
	\itemsep0cm
	\item Skala och klyfta äpplena
	\item Lägg hälften av frukten i en form
	\item Strö över ca 1 msk socker samt kanel
	\item Lägg på resten av frukten och strö över ytterligare 1 msk socker samt kanel
	\item Grädda frukten i 225$\degree$, ca 10 \mbox{minuter}
	\item Lägg på smuldegen
	\item Grädda frukten i 225$\degree$, ca 12 \mbox{minuter}
\end{enumerate}

\end{multicols*}

\clearpage

\subsection{Knäckäppelpaj}
\sats{4-6 portioner}

\begin{table}[H]
	\begin{tabular}{rl}
	% Mängd | Ingrediens (Ev. anmärkning)
	\hline
	&\\
		2-3 & \textbf{äpplen}\\
		150 g & \textbf{smör}\\
		3 dl & \textbf{havregryn}\\
		1$\frac{1}{2}$ dl & \textbf{socker}\\
		$\frac{1}{2}$ dl & \textbf{ljus sirap}\\
		1$\frac{1}{2}$ dl & \textbf{vetemjöl}\\
		$\frac{1}{2}$ tsk & \textbf{bakpulver}\\
		2 msk & \textbf{mjölk}\\
	&\\
	\hline
	\end{tabular}
\end{table}

\tillb{vaniljsås}

\begin{multicols*}{2}

\noindent \textbf{Tillagning}
\begin{enumerate}
	\itemsep0cm
	\item Smält smöret i en kastrull
	\item Blanda havregryn, socker, vetemjöl och bakpulver
	\item Skala och kärna ur äpplena, skär i tunna skivor
	\item Smörj en glasform och lägg i äpplena
	\item Tillsätt sirap, mjölk och smör till smeten
	\item Fördela smeten över äpplena
	\item Grädda i 175$\degree$, 20-30 \mbox{minuter}
\end{enumerate}

\end{multicols*}

\clearpage
\subsection{Schweizernötkaka}

\begin{table}[H]
	\begin{tabular}{rl}
	% Mängd | Ingrediens (Ev. anmärkning)
	\hline
	&\\
		2 & \textbf{ägg}\\
		3 dl & \textbf{socker}\\
		125 g & \textbf{sötmandel}\\
		1,25 dl & \textbf{vetemjöl}\\
		100 g & \textbf{smält smör}\\
		200 g & \textbf{schweizernötchoklad}\\
		1 dl & \textbf{vispgrädde}\\
		15 g& \textbf{smör}\\
	&\\
	\hline
	\end{tabular}
\end{table}

\tillb{vispgrädde}

\begin{multicols*}{2}

\noindent \textbf{Tillagning}
\begin{enumerate}
	\itemsep0cm
	\item Smält smöret.
	\item Skålla och skala mandeln.
	\item Mal mandeln med mandelkvarn.
	\item Vispa ägg och socker pösigt.
	\item Blanda ner den malda mandeln, mjölet och sist det smälta smöret.
	\item Häll smeten i smord rund kakform med löstagbar kant
	\item Grädda i mitten av ugnen i 175$\degree$, ca 25-30 \mbox{minuter} till lätt gyllenbrun färg.
	\item Låt kakan svalna något.
	\item Smält under tiden chokladen i en kastrull på spisen på låg värme.
	\item Rör ner grädden och sist smöret i chokladen.
	\item Låt chokladsmeten svalan något, men den ska fortfarande vara flytande.
	\item Bred chokladsmeten på kakan och ställ svalt.
\end{enumerate}

\end{multicols*}

\clearpage

%%%%%%%%%%%%%%%%%%%%%%%%%%%%%%%%%%%%%%
%-------Kakor och fikabröd-----------%
%%%%%%%%%%%%%%%%%%%%%%%%%%%%%%%%%%%%%%

\section{Kakor och fikabröd}

\clearpage

\subsection{Mattias kladdkaka}

\begin{table}[H]
	\begin{tabular}{rl}
	% Mängd | Ingrediens (Ev. anmärkning)
	\hline
	&\\
		2 & \textbf{ägg}\\
		3 dl & \textbf{socker}\\
		4 msk & \textbf{kakao}\\
		1$\frac{1}{2}$ dl & \textbf{vetemjöl}\\
		1 krm & \textbf{salt}\\
		150 g & \textbf{smör/margarin} (smält)\\
	&\\
	\hline
	\end{tabular}
\end{table}

\tillb{vispgrädde}

\begin{multicols*}{2}

\noindent \textbf{Tillagning}
\begin{enumerate}
	\itemsep0cm
	\item Smält smör
	\item Rör ihop (vispa ej) ägg och \mbox{socker}
	\item Tillsätt kakao, vetemjöl och salt
	\item Smörj och bröa form med löstagbar kant
	\item Tillsätt resten av smöret till smeten
	\item Grädda i 160$\degree$, 40 - 45 minuter
\end{enumerate}

\end{multicols*}

\clearpage

\subsection{Lingonkaka}

\begin{table}[H]
	\begin{tabular}{rl}
	% Mängd | Ingrediens (Ev. anmärkning)
	\hline
	&\\
		2 & \textbf{ägg}\\
		3 dl & \textbf{socker}\\
		2$\frac{1}{2}$ dl & \textbf{vetemjöl}\\
		$\frac{1}{2}$ tsk & \textbf{bakpulver}\\
		125 g & \textbf{smör/margarin} (smält, avsvalnat)\\
		3-4 dl & \textbf{lingon}\\
	&\\
	\hline
	\end{tabular}
\end{table}

\garn{florsocker}

\begin{multicols*}{2}

\noindent \textbf{Tillagning}
\begin{enumerate}
	\itemsep0cm
	\item Smält smör
	\item Vispa ägg och 2 dl socker pösigt
	\item Tillsätt vetemjöl och bakpulver
	\item Smörj och bröa form med löstagbar kant
	\item Tillsätt resten av smöret till smeten
	\item Häll hälften av smeten i formen
	\item Lägg på lingon och 1 dl socker
	\item Häll på resten av smeten
	\item Grädda i 175$\degree$, 50 minuter
\end{enumerate}

\end{multicols*}

\clearpage

\subsection{Äppelkaka}

\begin{table}[H]
	\begin{tabular}{rl}
	% Mängd | Ingrediens (Ev. anmärkning)
	\hline
	&\\
		2 & \textbf{ägg}\\
		3 dl & \textbf{socker}\\
		3 dl & \textbf{vetemjöl}\\
		100 g & \textbf{smör}\\
		3-4 & \textbf{äpplen}\\
		lite & \textbf{kanel}\\
	&\\
	\hline
	\end{tabular}
\end{table}

\begin{multicols*}{2}

\noindent \textbf{Tillagning}
\begin{enumerate}
	\itemsep0cm
	\item Smörj och bröa form med löstagbar kant
	\item Rör smör och socker pösigt
	\item Blanda i ägg och vetemjöl
	\item Häll smeten i formen
	\item Skala och skär äpplen i tunna klyftor
	\item Vänd klyftorna i socker och \mbox{kanel}, stick ner i smeten
	\item Grädda i 175$\degree$, ca 55 minuter
\end{enumerate}

\end{multicols*}

\clearpage

\subsection{Snittkakor}
\sats{ca 56 kakor}

\begin{table}[H]
	\begin{tabular}{rl}
	% Mängd | Ingrediens (Ev. anmärkning)
	\hline
	&\\
		200 g & \textbf{smör/margarin}\\
		2 dl & \textbf{strösocker}\\
		2 msk & \textbf{vanillinsocker}\\
		2 msk & \textbf{ljus sirap}\\
		2 tsk & \textbf{bakpulver}\\
		ca 5 dl & \textbf{vetemjöl}\\
	&\\
	\hline
	\end{tabular}
\end{table}

\begin{multicols*}{2}

\noindent \textbf{Tillagning}
\begin{enumerate}
	\itemsep0cm
	\item Rör smör och strösocker till en pösig smet
	\item Tillsätt vanillinsocker, sirap, bakpulver och mjöl
	\item Dela degen i fyra bitar och \mbox{platta} ut i längder på pappersklädd plåt
	\item Grädda i 200$\degree$, ca 15 minuter
	\item Skär längderna i smala bitar och låt svalna på plåten
\end{enumerate}

\end{multicols*}

\clearpage

\subsection{Vetebullar}
\sats{ca 40 bullar}

\begin{table}[H]
	\begin{tabular}{rl}
	% Mängd | Ingrediens (Ev. anmärkning)
	\hline
	&\\
		150 g & \textbf{smör/margarin}\\
		5 dl & \textbf{mjölk}\\
		50 g & \textbf{jäst}\\
		1 dl & \textbf{socker}\\
		$\frac{1}{2}$ tsk & \textbf{salt}\\
		2 tsk & \textbf{malen kardemumma}\\
		ca 14 dl & \textbf{vetemjöl}\\
		lite & \textbf{vaniljsocker} eller \textbf{kanel}\\
		1 & \textbf{ägg}\\
		lite & \textbf{pärlsocker}\\
	&\\
	\hline
	\end{tabular}
\end{table}

\begin{multicols*}{2}

\noindent \textbf{Tillagning}
\begin{enumerate}
	\itemsep0cm
	\item Smält smöret/margarinet
	\item Smula gästen i en bunke
	\item Blanda ner mjölken i marga\-rinet, värm till ca 37$\degree$
	\item Häll degspadet på jästen, lös jästen i degspadet
	\item	Blanda i socker, salt och kardemumma
	\item Tillsätt vetemjöl (spara 1 dl till utbakningen)
	\item Arbeta degen kraftigt (ca 10 minuter) tills smidig
	\item Låt degen jäsa i 40 minuter
	\item Knåda degen, dela i fyra delar och kavla ut till avlånga kakor
	\item Bred på margarin, strö på \mbox{socker} och sikta på vaniljsocker eller kanel
	\item Rulla och skär till bullar, sätt på plåt och låt jäsa i 30 \mbox{minuter}
	\item Pensla med ägg och lite mjölk, strö på pärlsocker
	\item Grädda i 250$\degree$, ca 10 minuter
	\item Låt svalna på galler under bakduk
\end{enumerate}

\end{multicols*}

\clearpage

\subsection{Mormor Monikas mjuka pepparkaka}

\begin{table}[H]
	\begin{tabular}{rl}
	% Mängd | Ingrediens (Ev. anmärkning)
	\hline
	&\\
		3 dl & \textbf{socker}\\
		3$\frac{3}{4}$ dl & \textbf{vetemjöl}\\
		2 msk & \textbf{hasselnötter} (malda)\\
		1 tsk & \textbf{kanel}\\
		1 tsk & \textbf{nejlikor} (malda)\\
		1$\frac{1}{2}$ tsk & \textbf{bikarbonat}\\
		3 dl & \textbf{filmjölk} (3$\%$ fetthalt, rumstempererad)\\
		50 g & \textbf{smör}\\
	&\\
	\hline
	\end{tabular}
\end{table}

\begin{multicols*}{2}

\noindent \textbf{Tillagning}
\begin{enumerate}
	\itemsep0cm
	\item Mal hasselnötterna
	\item Smält smöret
	\item Blanda alla torra ingredienser
	\item Tillsätt rumstempererad filmjölk och avsvalnat smör
	\item Häll smeten i en smord och \mbox{bröad} avlång form
	\item Grädda i 190$\degree$, ca 50 minuter
\end{enumerate}

\end{multicols*}

\clearpage

%%%%%%%%%%%%%%%%%%%%%%%%%%%%%%%%%%%%%%
%------------Bröd--------------------%
%%%%%%%%%%%%%%%%%%%%%%%%%%%%%%%%%%%%%%

\section{Bröd}

\clearpage

\subsection{Siktkakor}
\sats{4-6 kakor}

\begin{table}[H]
	\begin{tabular}{rl}
	% Mängd | Ingrediens (Ev. anmärkning)
	\hline
	&\\
		100 g & \textbf{jäst}\\
		50 g & \textbf{margarin}\\
		3$\frac{1}{3}$ dl & \textbf{vatten} \\
		1$\frac{1}{10}$ dl (160 g) & \textbf{mörk sirap}\\
		$\frac{1}{3}$ tsk & \textbf{salt}\\
		1 tsk & \textbf{ättikssprit} (outspädd)\\
		1 tsk & \textbf{bakpulver}\\
		5 dl & \textbf{rågsikt}\\
		5-7 dl & \textbf{vetemjöl}\\
		&\\
	\hline
	\end{tabular}
\end{table}

\begin{multicols*}{2}

\noindent \textbf{Tillagning}
\begin{enumerate}
	\itemsep0cm
	\item Smält margarinet
	\item Smula gästen i en bunke
	\item Blanda ner vatten i margarinet, värm till ca 37$\degree$
	\item Häll degspadet på jästen, lös jästen i degspadet
	\item	Blanda i sirap, salt och ättikssprit
	\item Blanda bakpulvret i rågsikten
	\item Tillsätt rågsikt och bakpulver
	\item Tillsätt vetemjöl
	\item Slå degen med trägaffel
	\item Låt degen jäsa i 25 minuter
	\item Kavla eller platta ut kakor på bakplåtspapper, nagga med gaffel
	\item Låt jäsa i 30 minuter
	\item Grädda i 225$\degree$, ca 10 minuter
\end{enumerate}

\end{multicols*}

\end{document}
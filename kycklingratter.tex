%%%%%%%%%%%%%%%%%%%%%%%%%%%%%%%%%%%%%%
%------------Kycklingrätter----------%
%%%%%%%%%%%%%%%%%%%%%%%%%%%%%%%%%%%%%%

\section{Kycklingrätter}

\clearpage

\subsection{Oreganokyckling}
\port{4}

\begin{table}[H]
	\begin{tabular}{rl}
	% Mängd | Ingrediens (Ev. anmärkning)
	\hline
	&\\
		4 & \textbf{kycklingfiléer}\\
		2 klyftor & \textbf{vitlök} (hackad)\\
		1 msk & \textbf{smör}\\
		$\frac{1}{2}$ tsk & \textbf{salt}\\
		1 krm & \textbf{svartpeppar}\\
		2$\frac{1}{2}$ dl & \textbf{matlagningsgrädde}\\
		1 tärning & \textbf{hönsbuljong}\\
		2-3 tsk & \textbf{oregano} (färsk eller torkad)\\
		1$\frac{1}{2}$ msk & \textbf{äppelcider-/balsamvinäger}\\
	&\\
	\hline
	\end{tabular}
\end{table}

\tillb{klyftpotatis}

\begin{multicols*}{2}

\noindent \textbf{Tillagning}
\begin{enumerate}
	\itemsep0cm
	\item Skär kycklingfiléerna på \mbox{längden} i lagom stora bitar
	\item Bryn kycklingen i stekpanna, salta och peppra
	\item Blanda grädde, buljong, äppelcider-/balsamvinäger och oregano i en gryta
	\item Tillsätt kycklingen och vitlöken
	\item Låt koka under lock ca 5 \mbox{minuter}
\end{enumerate}

\end{multicols*}

\clearpage

\subsection{Viltkyckling}
\port{4}

\begin{table}[H]
	\begin{tabular}{rl}
	% Mängd | Ingrediens (Ev. anmärkning)
	\hline
	&\\
		4 & \textbf{kycklingfiléer}\\
		3 msk & \textbf{margarin}\\
		1 krm & \textbf{svartpeppar}\\
		100 g & \textbf{champinjoner}\\
		2$\frac{1}{2}$ dl & \textbf{matlagningsgrädde}\\
		$\frac{1}{2}$ dl & \textbf{svartvinbärssaft}\\
		1 dl & \textbf{vatten}\\
		1$\frac{1}{2}$ msk & \textbf{viltfond}\\
		5 & \textbf{enbär}\\
		1 tsk & \textbf{timjan}\\
		1 msk & \textbf{soja}\\
		1$\frac{1}{2}$ msk & \textbf{vetemjöl}\\
	&\\
	\hline
	\end{tabular}
\end{table}

\tillb{klyftpotatis}

\begin{multicols*}{2}

\noindent \textbf{Tillagning}
\begin{enumerate}
	\itemsep0cm
	\item Skär kycklingfiléerna på \mbox{längden} i lagom stora bitar
	\item Bryn kycklingen i stekpanna, salta och peppra
	\item Skiva champinjonerna och stek i margarin
	\item Rör ner vetemjöl
	\item Blanda i grädde, saft, vatten, fond, enbär, timjan och soja
	\item Tillsätt kycklingen
	\item Låt koka under lock ca 10 \mbox{minuter}
	\item Smaka av med salt och peppar
\end{enumerate}

\end{multicols*}

\clearpage

\subsection{Kycklingwok}
\port{ca 8}

\begin{table}[H]
	\begin{tabular}{rl}
	% Mängd | Ingrediens (Ev. anmärkning)
	\hline
	&\\
		1 kg & \textbf{kycklingkött}\\
		2$\frac{2}{3}$ msk & \textbf{soja}\\
		4 tsk & \textbf{ingefära} (riven)\\
		4 klyftor & \textbf{vitlök} (pressad)\\
		2 & \textbf{paprikor}\\
		2 - 3 & \textbf{palsternackor}\\
		2 - 3 & \textbf{morötter}\\
		150 g (ca 30 cm) & \textbf{purjolök}\\
		2$\frac{1}{2}$ msk & \textbf{sesamfrö}\\
		2 msk & \textbf{sweet chilisås}\\
		2 krm & \textbf{salt}\\
	&\\
	\hline
	\end{tabular}
\end{table}

\tillb{ris}

\begin{multicols*}{2}

\noindent \textbf{Tillagning}
\begin{enumerate}
	\itemsep0cm
	\item Skär kycklingköttet i strimlor eller små bitar
	\item Blanda soja, ingefära och vitlök och häll blandingen över köttet
	\item Skala morötter och palster\-nackor, skär i tunna strimlor
	\item Rensa paprikan och skär i \mbox{tunna} bitar
	\item Skär purjolöken i tunna skivor
	\item Rosta sesamfröna i stekpanna på hög värme (utan matfett) tills de fått lite färg, rör hela \mbox{tiden}
	\item Woka köttet tills det är genomstekt och flytta det sedan till stekgryta
	\item Woka palsternackan, \mbox{morötterna}, purjolöken samt paprikan och blanda med kötttet
	\item Tillsätt salt, sweet chilisås och de rostade sesamfröna
\end{enumerate}

\end{multicols*}

\clearpage

\subsection{Flygande Jacob}
\port{6}

\begin{table}[H]
	\begin{tabular}{rl}
	% Mängd | Ingrediens (Ev. anmärkning)
	\hline
	&\\
		800-900 g & \textbf{kycklingkött}\\
		140 g (1 paket) & \textbf{bacon}\\
		1-2 & \textbf{bananer}\\
		1 dl & \textbf{jordnötter}\\
		2 dl & \textbf{vispgrädde}\\
		3 msk & \textbf{chilisås}\\
	&\\
	\hline
	\end{tabular}
\end{table}

\tillb{ris}

\begin{multicols*}{2}

\noindent \textbf{Tillagning}
\begin{enumerate}
	\itemsep0cm
	\item Stek kycklingen i ugn, 225$\degree$, ca 25 minuter
	\item Skär kycklingen i mindre bitar
	\item Klipp bacon i mindre bitar och stek i stekpanna
	\item Lägg kyckling och bacon i en ugnsfrom, strö över \mbox{jordnötterna}
	\item Skiva bananerna, lägg i formen eller servera färsk som tillbehör
	\item Vispa grädden relativt hårt
	\item Blanda ner chilisåsen i grädden
	\item Bre ut gräddblandningen i \mbox{formen}
	\item Gratinera i ugnen 225$\degree$, ca 10 minuter
\end{enumerate}

\end{multicols*}

\clearpage